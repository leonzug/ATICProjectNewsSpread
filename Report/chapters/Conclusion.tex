\section{Conclusion}
We started this work by motivating why the study of information spread in a society is so important, and came up with a simplified society where such spread could be studied. Here, the influence of several personality traits of the population's individuals, and the presence of real and fake news sources was studied. Our model has a linear update equation, which allows to use a lot of theoretical tools for analysis purposes. The model has a flexible implementation, which allows to test many different scenarios and even to make the society more complex by adding personality traits.
Numerical experiments were used to validate our model. In Section~\ref{sec:Instr_Level}, the average instruction level is varied and it is shown that this helps mitigate the spread of fare news. In Section~\ref{sec:diversity}, it is shown that in a diverse society, a broader spectrum of ideas have a place to exist. In a society where individuals are easily manipulable, as shown in Section~\ref{sec:manipulability}, ideas spread more rapidly than if the individuals are less manipulable.
Finally, in Section~\ref{sec:polarization} we propose a slightly modified model where we build a society in such a way that it is divided in two main groups, and show that this leads to polarization of ideas.\\
As mentioned, our model is simple but powerful, as it is built in such a way that it is expandable. Future work could focus on including new personality traits, or modifying how the weights are generated based on the defined personality traits. Another realistic addition to the model could be to give a different strength to news sources, i.e. make them more or less connected and with larger or smaller weights connecting them to the individuals.
Furthermore, focus  could be put on developing the algorithm that creates connections between individuals, to create different community structures. Finally, the model could be extended to include also non-constant adjacency matrices to capture more effects of the opinion spread, such as perturbations or active control on the system. 
 