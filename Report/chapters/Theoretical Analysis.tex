\section{Theoretical Analysis}
\subsection{General Considerations}
Without loss of generality, the labels of the nodes in the graph can be selected so that the indices referring to the fake/real news sources appear to be the last ones. Considering $n_{People}$ as the number of people in the network and $n_{Sources}$ as the number of fake and real news sources, the matrix A can be partitioned as illustrated in the following:

$$
(1)\ A = 
\begin{bmatrix}
	\tilde{A}_{n_{People} \times n_{People}} & \tilde{B}_{n_{People} \times n_{Sources}} \\
	0_{n_{Sources} \times n_{People}} & I_{n_{Sources} \times n_{Sources}} \\
\end{bmatrix} 
\quad
$$

Where $\tilde{A}$ describes the reciprocal human-human interactions and $\tilde{B}$ describes how people's opinions are directly affected by the sources. The zero and identity matrices below encode the fact that the sources' opinions are not changing over time. \newline
The matrix A is row-stochastic and this implies that the vector of all ones is eigenvector of A with eigenvalue 1. The Gershgorin disk theorem tells that a row-stochastic matrix cannot have an eigenvalue with magnitude greater than one. By applying the theorem to our specific case, where each diagonal element is different from zero, it can be concluded that the only possible eigenvalue with magnitude equal to one is 1 itself. \newline
$$
spec(A) \subset \{1,\ \mu \in \mathbb{R},\ | \mu | < 1\}
$$
Because $\rho(A) = 1$, the matrix A cannot be stable. The matrix A is semi-stable iff $1 \in spec(A)$ is semi-simple. \newline
The randomness in the graph generation process do not allow to conclude much more deterministically. The properties of $\tilde{A}$ are of great importance to infer the system behavior. It is remarkable that with our set up: N = 100 individuals and C = 0.2, nRoot = 4 (this fully describes the probabilistic model of the connections between a person and his neighbors, see previous chapter), the $\tilde{A}$ describes a strongly-connected graph in 99.4$\%$ of the cases. This value was obtained after generating 10000 different networks. The graph is directed and a connection from node i to node j implies a connection from j to i but with different weights in general. The 0.6$\%$ represents therefore some pathological and unlikely cases where part of the network is completely disconnected from the rest. This could represent an isolated and separated society living in our simulated world. Since we are not interested in such cases (we want to understand how news spread in a connected society) we simply discarded these cases from our experiments. From now on, it can therefore be assumed that the graph describing the human-human interactions is strongly connected, meaning that the matrix $\tilde{A}$ is irreducible. In the following, the steady state behavior of the discrete time averaging system 
$$
x(t+1) = Ax(t)
$$
is analyzed under various circumstances.
\subsection{Without Sources}
If there are no sources, the matrix A = $\tilde{A}$ describes a strongly connected digraph. Each node has a self-cycle and this implies that the digraph is acyclic. The graph is therefore strongly connected and acyclic and this implies that the matrix describing it is primitive. This, together with the fact that A is row-stochastic, implies that A is semi-convergent and that the discrete-time averaging system convergences to consensus, namely:
$$
x_{\infty} = (w^Tx(0))1_n = (\sum_{i=1}^{n_{People}}w_ix_i(0))1_n
$$
$$
w^TA = w^T,\ 
Av = v,\ 
v \geq 0,\ w \geq 0,\ w^Tv = 1
$$
\subsection{With Sources}


